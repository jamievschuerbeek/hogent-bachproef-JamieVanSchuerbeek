%==============================================================================
% Sjabloon onderzoeksvoorstel bachproef
%==============================================================================
% Gebaseerd op document class `hogent-article'
% zie <https://github.com/HoGentTIN/latex-hogent-article>

% Voor een voorstel in het Engels: voeg de documentclass-optie [english] toe.
% Let op: kan enkel na toestemming van de bachelorproefcoördinator!
\documentclass{hogent-article}

% Invoegen bibliografiebestand
\addbibresource{voorstel.bib}

% Informatie over de opleiding, het vak en soort opdracht
\studyprogramme{Professionele bachelor toegepaste informatica}
\course{Bachelorproef}
\assignmenttype{Onderzoeksvoorstel}
% Voor een voorstel in het Engels, haal de volgende 3 regels uit commentaar
% \studyprogramme{Bachelor of applied information technology}
% \course{Bachelor thesis}
% \assignmenttype{Research proposal}

\academicyear{2023-2024} % TODO: pas het academiejaar aan

% TODO: Werktitel
\title{Manieren om master data uit bedrijfsdocumenten te halen}

% TODO: Studentnaam en emailadres invullen
\author{Jamie Van Schuerbeek}
\email{jamie.vanschuerbeek@student.hogent.be}

% TODO: Medestudent
% Gaat het om een bachelorproef in samenwerking met een student in een andere
% opleiding? Geef dan de naam en emailadres hier
% \author{Yasmine Alaoui (naam opleiding)}
% \email{yasmine.alaoui@student.hogent.be}

% TODO: Geef de co-promotor op
\supervisor[Co-promotoren]{A. Jacobs en N. De Greef (Avelon, \href{mailto:alexander.jacobs@avelon.be}{alexander.jacobs@avelon.be}, \href{mailto:niels.degreef@avelon.be}{niels.degreef@avelon.be})}

% Binnen welke specialisatierichting uit 3TI situeert dit onderzoek zich?
% Kies uit deze lijst:
%
% - Mobile \& Enterprise development
% - AI \& Data Engineering
% - Functional \& Business Analysis
% - System \& Network Administrator
% - Mainframe Expert
% - Als het onderzoek niet past binnen een van deze domeinen specifieer je deze
%   zelf
%
\specialisation{Functional \& Business Analysis}
\keywords{SAP, Master Data Management, Document Information Extraction}

\begin{document}

\begin{abstract}
%  Hier schrijf je de samenvatting van je voorstel, als een doorlopende tekst van één paragraaf. Let op: dit is geen inleiding, maar een samenvattende tekst van heel je voorstel met inleiding (voorstelling, kaderen thema), probleemstelling en centrale onderzoeksvraag, onderzoeksdoelstelling (wat zie je als het concrete resultaat van je bachelorproef?), voorgestelde methodologie, verwachte resultaten en meerwaarde van dit onderzoek (wat heeft de doelgroep aan het resultaat?).

Avelon is een bedrijf dat software as a service- (SaaS) en eigen SAP softwareoplossingen biedt alsook consulting met betrekking tot SAP master data kwaliteit en bedrijfsprocess optimalisatie. In het dynamische landschap van de moderne bedrijfsvoering, is het efficiënt master data uit omvangrijke bedrijfsdocumenten halen cruciaal. In de meeste organisaties wordt dit nog steeds beschouwd als een overbodige handmatige activiteit. Masterdata, de kerninformatie die de belangrijkste entiteiten van een organisatie definieert zoals klanten, producten en leveranciers, speelt een belangrijke rol bij geïnformeerde besluitvorming maken en strategische planning. Avelon zou graag met dit onderzoek de verschillende manieren om dit process te automatiseren willen ontdekken. Om dit te realiseren zal er een analyse uitgevoerd worden naar verschillende machine learning(ML) diensten om informatie uit bedrijfsdocumenten te halen. Een paar voorbeelden hiervan zijn: Document Information Extraction (DOX) \& Business Entity Recognition (BER) op het SAP Business Technology Platform(BTP). Nadien zal er een proof of concept(POC) gebouwd worden die geïntegreerd wordt op het SAP Master Data Governance(MDG) Platform. Het verwachte resultaat van dit onderzoek is een werkend prototype van een mogelijk product te hebben dat Avelon kan gebruiken om verdere verbeteringen aan te maken om op die manier een volledig product te bekomen.
\end{abstract}

\tableofcontents

% De hoofdtekst van het voorstel zit in een apart bestand, zodat het makkelijk
% kan opgenomen worden in de bijlagen van de bachelorproef zelf.
%---------- Inleiding ---------------------------------------------------------

\section{Introductie}%
\label{sec:introductie}

%Waarover zal je bachelorproef gaan? Introduceer het thema en zorg dat volgende zaken zeker duidelijk aanwezig zijn:
%
%\begin{itemize}
%  \item kaderen thema
%  \item de doelgroep
%  \item de probleemstelling en (centrale) onderzoeksvraag
%  \item de onderzoeksdoelstelling
%\end{itemize}
%
%Denk er aan: een typische bachelorproef is \textit{toegepast onderzoek}, wat betekent dat je start vanuit een concrete probleemsituatie in bedrijfscontext, een \textbf{casus}. Het is belangrijk om je onderwerp goed af te bakenen: je gaat voor die \textit{ene specifieke probleemsituatie} op zoek naar een goede oplossing, op basis van de huidige kennis in het vakgebied.
%
%De doelgroep moet ook concreet en duidelijk zijn, dus geen algemene of vaag gedefinieerde groepen zoals \emph{bedrijven}, \emph{developers}, \emph{Vlamingen}, enz. Je richt je in elk geval op it-professionals, een bachelorproef is geen populariserende tekst. Eén specifiek bedrijf (die te maken hebben met een concrete probleemsituatie) is dus beter dan \emph{bedrijven} in het algemeen.
%
%Formuleer duidelijk de onderzoeksvraag! De begeleiders lezen nog steeds te veel voorstellen waarin we geen onderzoeksvraag terugvinden.
%
%Schrijf ook iets over de doelstelling. Wat zie je als het concrete eindresultaat van je onderzoek, naast de uitgeschreven scriptie? Is het een proof-of-concept, een rapport met aanbevelingen, \ldots Met welk eindresultaat kan je je bachelorproef als een succes beschouwen?

In deze tijd is data niet meer weg te denken en is het een van de belangrijkste zaken in de bedrijfswereld, een bedrijf dat dan ook efficiënt omgaat met zijn master data kan op die manier meer geïnformeerde beslissingen maken. Omdat het ophalen van interessante data uit bedrijfsdocumenten voor veel bedrijven nog een manuele taak blijft, is een oplossing die meer geautomatiseerd is steeds interessanter aan het worden.  De bedoeling van dit onderzoek is manieren te vinden om relevante informatie uit bedrijfsdocumenten te halen, nadien een werkend prototype te maken voor deze oplossing en dit te integreren in het SAP Master Data Governance platform om op deze manier master data records aan te maken. Op deze manier kan met behulp van machine learning de anderzijds manuele taak van informatie extractie geautomatiseerd worden.

%---------- Stand van zaken ---------------------------------------------------

\section{State-of-the-art}%
\label{sec:state-of-the-art}

%Hier beschrijf je de \emph{state-of-the-art} rondom je gekozen onderzoeksdomein, d.w.z.\ een inleidende, doorlopende tekst over het onderzoeksdomein van je bachelorproef. Je steunt daarbij heel sterk op de professionele \emph{vakliteratuur}, en niet zozeer op populariserende teksten voor een breed publiek. Wat is de huidige stand van zaken in dit domein, en wat zijn nog eventuele open vragen (die misschien de aanleiding waren tot je onderzoeksvraag!)?
%
%Je mag de titel van deze sectie ook aanpassen (literatuurstudie, stand van zaken, enz.). Zijn er al gelijkaardige onderzoeken gevoerd? Wat concluderen ze? Wat is het verschil met jouw onderzoek?
%
%Verwijs bij elke introductie van een term of bewering over het domein naar de vakliteratuur, bijvoorbeeld~\autocite{Hykes2013}! Denk zeker goed na welke werken je refereert en waarom.
%
%Draag zorg voor correcte literatuurverwijzingen! Een bronvermelding hoort thuis \emph{binnen} de zin waar je je op die bron baseert, dus niet er buiten! Maak meteen een verwijzing als je gebruik maakt van een bron. Doe dit dus \emph{niet} aan het einde van een lange paragraaf. Baseer nooit teveel aansluitende tekst op eenzelfde bron.
%
%Als je informatie over bronnen verzamelt in JabRef, zorg er dan voor dat alle nodige info aanwezig is om de bron terug te vinden (zoals uitvoerig besproken in de lessen Research Methods).
%
%% Voor literatuurverwijzingen zijn er twee belangrijke commando's:
%% \autocite{KEY} => (Auteur, jaartal) Gebruik dit als de naam van de auteur
%%   geen onderdeel is van de zin.
%% \textcite{KEY} => Auteur (jaartal)  Gebruik dit als de auteursnaam wel een
%%   functie heeft in de zin (bv. ``Uit onderzoek door Doll & Hill (1954) bleek
%%   ...'')
%
%Je mag deze sectie nog verder onderverdelen in subsecties als dit de structuur van de tekst kan verduidelijken.

Information Extraction(IE) is een manier om tekst te doorzoeken met als doel relevante informatie voor de belanghebbende te vinden. Het omvangt meer dan enkel en alleen trefwoorden te zoeken in de tekst, maar de doelstellingen blijven achter bij het probleem van tekstbegrip, waarbij we proberen alle informatie in een tekst vast te leggen, samen met de intentie van de schrijver \autocite{hobbs2010}. Vroegere IE systemen waren meer gelimiteerd en konden namen, relaties en gebeurtenissen uit simpele teksten halen die een vaste structuur hadden, zoals ``locatie werd gebombardeerd'' en ``persoon werd aangenomen''. De nieuwste systemen van tegenwoordig kunnen door machine learning veel meer zaken herkennen waaronder meer domeinspecifieke zaken zoals ziekten, wetten en wetenschappelijke resultaten \autocite{Small2014}.

IE ligt tussen twee andere methodes van tekstverwerking, namelijk information retrieval en natuurlijke taalverwerking. IE kan worden beschouwd als een soort goedkoop, gericht begrijpen van natuurlijke taal. IE gaat uit van een verzameling documenten, waarin elk document namen en gebeurtenissen beschrijft die op elkaar lijken, maar waar de details verschillen. Voor een IE-taak wordt er een sjabloon voorzien die beschrijft wat voor informatie er in het document staat \autocite{Freitag2000}

Document information extraction(DOX) is een dienst aangeboden door SAP die het mogelijk maakt om met machine learning informatie uit documenten te halen. Op deze manier wordt het verwerken van grote hoeveelheden aan documenten die hun inhoud in titels en tabellen hebben verwerkt. Dit geeft de mogelijkheid om documenten zoals facturen en betalingsdocumenten automatisch te verwerken \autocite{SAPDOX}.

Business entity recognition(BER) is een gelijkaardige dienst die wordt aangeboden door SAP. Het grootste verschil is dat BER het mogelijk maakt om informatie uit ongestructureerde stukken tekst te halen. Dit biedt de mogelijkheid om bijvoorbeeld de context uit e-mails te halen of het automatiseren van herhalende taken zoals het beantwoorden van vragen over de status en betaling van facturen. Hier kun je voorgetrainde modellen gebruiken of eigen modellen om ervoor te zorgen dat de resultaten geschikter zijn voor de doeleinden van de gebruiker. Deze toepassing kan helpen om het manuele en herhalende werk in een bedrijf te verminderen zodat werknemers meer tijd hebben om zich te focussen op belangrijkere zaken.
 \autocite{SAPBER}
%---------- Methodologie ------------------------------------------------------
\section{Methodologie}%
\label{sec:methodologie}

%Hier beschrijf je hoe je van plan bent het onderzoek te voeren. Welke onderzoekstechniek ga je toepassen om elk van je onderzoeksvragen te beantwoorden? Gebruik je hiervoor literatuurstudie, interviews met belanghebbenden (bv.~voor requirements-analyse), experimenten, simulaties, vergelijkende studie, risico-analyse, PoC, \ldots?
%
%Valt je onderwerp onder één van de typische soorten bachelorproeven die besproken zijn in de lessen Research Methods (bv.\ vergelijkende studie of risico-analyse)? Zorg er dan ook voor dat we duidelijk de verschillende stappen terug vinden die we verwachten in dit soort onderzoek!
%
%Vermijd onderzoekstechnieken die geen objectieve, meetbare resultaten kunnen opleveren. Enquêtes, bijvoorbeeld, zijn voor een bachelorproef informatica meestal \textbf{niet geschikt}. De antwoorden zijn eerder meningen dan feiten en in de praktijk blijkt het ook bijzonder moeilijk om voldoende respondenten te vinden. Studenten die een enquête willen voeren, hebben meestal ook geen goede definitie van de populatie, waardoor ook niet kan aangetoond worden dat eventuele resultaten representatief zijn.
%
%Uit dit onderdeel moet duidelijk naar voor komen dat je bachelorproef ook technisch voldoen\-de diepgang zal bevatten. Het zou niet kloppen als een bachelorproef informatica ook door bv.\ een student marketing zou kunnen uitgevoerd worden.
%
%Je beschrijft ook al welke tools (hardware, software, diensten, \ldots) je denkt hiervoor te gebruiken of te ontwikkelen.
%
%Probeer ook een tijdschatting te maken. Hoe lang zal je met elke fase van je onderzoek bezig zijn en wat zijn de concrete \emph{deliverables} in elke fase?

Het onderzoek zal voornamelijk uit 3 grote delen bestaan.
Het eerste deel zal een grote analyse zijn van een tweetal weken, met als doel goede kennis te krijgen over het onderwerp. Dit zal bestaan uit een studie van de verschillende methodes om informatie uit documenten te halen. Hier wordt algemene kennis over de verschillende methodes opgedaan over wat ze precies zijn en hoe ze werken. Later zal dit toegepast worden op de diensten en hulpmiddelen die ter beschikking worden gesteld om dergelijke toepassingen te maken. Er wordt dan een uitgebreide studie gedaan van de documentatie en artikelen van professionals in het vakgebied. Er wordt tijdens deze studie naar een aantal verschillende parameters gekeken, namelijk: prijs, gebruiksgemak, kwaliteit en de mogelijkheid tot integratie met SAP platformen.

Als het eerste deel afgerond is zal er in het volgende deel eerst een aantal experimenten volgen om de theorie van het eerste deel te kunnen testen in de praktijk. In deze fase wordt vooral naar gebruiksgemak en kwaliteit gekeken. Er zal een aantal testen opgezet worden die verschillende diensten uitproberen, om zo te kunnen kijken met welke service gemakkelijk een document extractie-service kan gebouwd worden die kwalitatieve master data levert. Uit de resultaten van deze testen zal een service gekozen worden waarmee een concrete toepassing gebouwd kan worden in de vorm van een proof of concept(PoC). 

In het laatste deel van dit onderzoek is het hoofdzakelijk de bedoeling om de PoC te implementeren in het SAP master data governance platform. Op deze manier kan de master data die uit documenten gehaald wordt direct in SAP gebruikt worden. Nadien zal de toepassing uitermate getest en gebenchmarked worden om de verschillende prestaties van de toepassing te krijgen zoals snelheid en kwaliteit van de data. Op deze manier kan er een uitgebreide conclusie te kunnen maken met een antwoord op de onderzoeksvraag en prestaties van de toepassing.

%---------- Verwachte resultaten ----------------------------------------------
\section{Verwacht resultaat, conclusie}%
\label{sec:verwachte_resultaten}

%Hier beschrijf je welke resultaten je verwacht. Als je metingen en simulaties uitvoert, kan je hier al mock-ups maken van de grafieken samen met de verwachte conclusies. Benoem zeker al je assen en de onderdelen van de grafiek die je gaat gebruiken. Dit zorgt ervoor dat je concreet weet welk soort data je moet verzamelen en hoe je die moet meten.
%
%Wat heeft de doelgroep van je onderzoek aan het resultaat? Op welke manier zorgt jouw bachelorproef voor een meerwaarde?
%
%Hier beschrijf je wat je verwacht uit je onderzoek, met de motivatie waarom. Het is \textbf{niet} erg indien uit je onderzoek andere resultaten en conclusies vloeien dan dat je hier beschrijft: het is dan juist interessant om te onderzoeken waarom jouw hypothesen niet overeenkomen met de resultaten.

Het verwachte resultaat op het einde van het onderzoek is een antwoord op de onderzoeksvraag doormiddel van een werkend proof of concept van een applicatie die gebruikt kan worden voor verschillende doeleinden in een SAP-systeem. Deze proof of concept zou documenten als input moeten kunnen krijgen en de applicatie zou correcte info moeten teruggeven met betrekking tot de toepassing waar het gebruikt wordt. Er zijn verschillende tools en diensten beschikbaar om het ontwikkelen van zulke applicaties mogelijk en zelfs gemakkelijker te maken. De snelheid waarop het ophalen van info gebeurt is in dit onderzoek minder van belang, vooral de correctheid en relevantie van de data is de grootste prioriteit die gelegd wordt.


\printbibliography[heading=bibintoc]

\end{document}