%%=============================================================================
%% Samenvatting
%%=============================================================================

% TODO: De "abstract" of samenvatting is een kernachtige (~ 1 blz. voor een
% thesis) synthese van het document.
%
% Een goede abstract biedt een kernachtig antwoord op volgende vragen:
%
% 1. Waarover gaat de bachelorproef?
% 2. Waarom heb je er over geschreven?
% 3. Hoe heb je het onderzoek uitgevoerd?
% 4. Wat waren de resultaten? Wat blijkt uit je onderzoek?
% 5. Wat betekenen je resultaten? Wat is de relevantie voor het werkveld?
%
% Daarom bestaat een abstract uit volgende componenten:
%
% - inleiding + kaderen thema
% - probleemstelling
% - (centrale) onderzoeksvraag
% - onderzoeksdoelstelling
% - methodologie
% - resultaten (beperk tot de belangrijkste, relevant voor de onderzoeksvraag)
% - conclusies, aanbevelingen, beperkingen
%
% LET OP! Een samenvatting is GEEN voorwoord!

%%---------- Nederlandse samenvatting -----------------------------------------
%
% TODO: Als je je bachelorproef in het Engels schrijft, moet je eerst een
% Nederlandse samenvatting invoegen. Haal daarvoor onderstaande code uit
% commentaar.
% Wie zijn bachelorproef in het Nederlands schrijft, kan dit negeren, de inhoud
% wordt niet in het document ingevoegd.

\IfLanguageName{english}{%
\selectlanguage{dutch}
\chapter*{Samenvatting}
\lipsum[1-4]
\selectlanguage{english}
}{}

%%---------- Samenvatting -----------------------------------------------------
% De samenvatting in de hoofdtaal van het document

\chapter*{\IfLanguageName{dutch}{Samenvatting}{Abstract}}
Alluvion is een bedrijf dat software as a service- (SaaS) en eigen SAP softwareoplossingen biedt alsook consulting met betrekking tot SAP master data kwaliteit en bedrijfsprocess optimalisatie.

In het dynamische landschap van de moderne bedrijfsvoering, is het efficiënt master data uit omvangrijke bedrijfsdocumenten halen cruciaal. In de meeste bedrijven wordt dit nog steeds beschouwd als een overbodige handmatige activiteit. Masterdata, de kerninformatie die de belangrijkste entiteiten van een organisatie definieert zoals klanten, producten en leveranciers, speelt een belangrijke rol bij geïnformeerde besluitvorming en strategische planning. Hoe kan machine learning-technologie worden toegepast om relevante masterdata uit bedrijfsdocumenten te halen? Met dit onderzoek zou Alluvion graag een antwoord op deze vraag willen krijgen door de verschillende manieren om dit proces te automatiseren te ontdekken.

Om dit doel te bereiken begint deze studie met een onderzoek naar deep learning technieken zoals, Natural Language Processing (NLP) en Information Extraction (IE), die kunnen gebruikt worden om dit doel te bereiken. Vervolgens wordt er ook onderzoek gedaan naar bestaande tools die deze technieken implementeren, zoals SAP Business Entity Recognition (BER). Deze tools worden vervolgens geëvalueerd op basis van verschillende factoren zoals prijs, kwaliteit, integratie met SAP platformen en gebruiksvriendelijkheid om zo een tool te kunnen bepalen waarmee een proof of concept wordt gemaakt.

Een praktische implementatie in de vorm van een proof of concept toont aan hoe de SAP BER service gebruikt kan worden om een praktische toepassing te maken die masterdata uit bedrijfsdocumenten kan halen. Dit proof of concept dient als een basis voor verdere ontwikkeling van een product dat Alluvion kan gebruiken om verdere verbeteringen aan te maken om op die manier een volledig product te bekomen. SAP BER samen gepaard met SAP Fiori vormt een eenvoudige manier om applicaties te bouwen die kunnen helpen bij het automatiseren van masterdata extractie uit bedrijfsdocumenten.

