%%=============================================================================
%% Voorwoord
%%=============================================================================

\chapter*{\IfLanguageName{dutch}{Woord vooraf}{Preface}}%
\label{ch:voorwoord}

%% TODO:
%% Het voorwoord is het enige deel van de bachelorproef waar je vanuit je
%% eigen standpunt (``ik-vorm'') mag schrijven. Je kan hier bv. motiveren
%% waarom jij het onderwerp wil bespreken.
%% Vergeet ook niet te bedanken wie je geholpen/gesteund/... heeft

Met veel trots presenteer ik u mijn bachelorproef, die het sluitstuk van mijn opleiding vormt en het resultaat is van lang onderzoek in een onderwerp waar ik daarvoor nog niet veel kennis over had. Machine Learning is een onderwerp dat mij erg interesseert, maar waar ik bijna geen ervaring had. Toen ik Alluvion contacteerde, met de vraag of zij een onderwerp hadden voor mijn bachelorproef, kwamen zij met het idee om een onderzoek te doen naar het gebruik van Machine Learning om bedrijfsdocumenten te analyseren. 

Ik wil daarom graag Alluvion en in het bijzonder mijn co-promotor, Alexander Jacobs, enorm bedanken voor de hulp en ondersteuning die zij mij geboden hebben voor het schrijven van deze bachelorproef. Hij heeft mij erg goed op weg kunnen helpen met het voorstellen van dit onderwerp en het geven van zeer waardevolle tips. Daarnaast wil ik ook mijn promotor, Chloé De Leenheer, bedanken voor de steun en feedback waar ik telkens op kon rekenen gedurende dit onderzoek. Tot slot wil ik ook mijn familie en vrienden bedanken voor de steun die zij mij gegeven hebben tijdens het schrijven van deze bachelorproef en gedurende mijn opleiding.