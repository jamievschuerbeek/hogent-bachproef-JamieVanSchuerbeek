%%=============================================================================
%% Inleiding
%%=============================================================================

\chapter{\IfLanguageName{dutch}{Inleiding}{Introduction}}%
\label{ch:inleiding}

%De inleiding moet de lezer net genoeg informatie verschaffen om het onderwerp te begrijpen en in te zien waarom de onderzoeksvraag de moeite waard is om te onderzoeken. In de inleiding ga je literatuurverwijzingen beperken, zodat de tekst vlot leesbaar blijft. Je kan de inleiding verder onderverdelen in secties als dit de tekst verduidelijkt. Zaken die aan bod kunnen komen in de inleiding~\autocite{Pollefliet2011}:

% \begin{itemize}
%   \item context, achtergrond
%   \item afbakenen van het onderwerp
%   \item verantwoording van het onderwerp, methodologie
%   \item probleemstelling
%   \item onderzoeksdoelstelling
%   \item onderzoeksvraag
%   \item \ldots
% \end{itemize}

In deze tijd is data niet meer weg te denken en is het een van de belangrijkste zaken in de bedrijfswereld, een bedrijf dat dan ook efficiënt omgaat met zijn master data kan op die manier meer geïnformeerde beslissingen maken. Omdat het ophalen van interessante data uit bedrijfsdocumenten voor veel bedrijven nog een manuele taak blijft, is een oplossing die meer geautomatiseerd is steeds interessanter aan het worden. De bedoeling van dit onderzoek is manieren te vinden om relevante informatie uit bedrijfsdocumenten te halen, nadien een werkend prototype te maken voor deze oplossing en dit te integreren in het SAP Master Data Governance platform om op deze manier master data records aan te maken. Op deze manier kan met behulp van machine learning de anderzijds manuele taak van informatie extractie geautomatiseerd worden.

\section{\IfLanguageName{dutch}{Probleemstelling}{Problem Statement}}%
\label{sec:probleemstelling}

Het gebruik van machine learning-technologie wordt tegenwoordig steeds meer toegepast. Maar hoe kan deze technologie worden gebruikt om relevante masterdata uit bedrijfsdocumenten te halen? Alluvion, een bedrijf gespecialiseerd in het gebruik van SAP Master Data management, wil graag weten hoe deze technologie gebruikt kan worden om deze, vaak nog manuele, taak te automatiseren. Door onderzoek te doen naar bestaande technologieën en tools en het maken van een proof of concept, kan Alluvion een beter beeld krijgen in de mogelijkheden en werking van deze technologieën.
\section{\IfLanguageName{dutch}{Onderzoeksvraag}{Research question}}%
\label{sec:onderzoeksvraag}
% Wees zo concreet mogelijk bij het formuleren van je onderzoeksvraag. Een onderzoeksvraag is trouwens iets waar nog niemand op dit moment een antwoord heeft (voor zover je kan nagaan). Het opzoeken van bestaande informatie (bv. ``welke tools bestaan er voor deze toepassing?'') is dus geen onderzoeksvraag. Je kan de onderzoeksvraag verder specifiëren in deelvragen. Bv.~als je onderzoek gaat over performantiemetingen, dan 
Dit onderzoek bevat een hoofd onderzoeksvraag:
Hoe kan machine learning-technologie worden toegepast om relevante masterdata uit bedrijfsdocumenten te halen?

Verder zijn er nog een deelvraag:

Welke tools bestaan er die gebruikt kunnen worden een dergelijke oplossing te maken?

\section{\IfLanguageName{dutch}{Onderzoeksdoelstelling}{Research objective}}%
\label{sec:onderzoeksdoelstelling}

%Wat is het beoogde resultaat van je bachelorproef? Wat zijn de criteria voor succes? Beschrijf die zo concreet mogelijk. Gaat het bv.\ om een proof-of-concept, een prototype, een verslag met aanbevelingen, een vergelijkende studie, enz.
De doelstelling van dit onderzoek is door een literatuurstudie een inzicht te krijgen over de verschillende onderwerpen die relevant zijn in dit onderzoek. Vervolgens is het de bedoeling om een proof of concept te maken die aantoont hoe de kennis uit de literatuuurstudie kan worden toegepast om een applicatie te maken die dient als basis voor een dienst die verder uitgebreid kan worden.  

\section{\IfLanguageName{dutch}{Opzet van deze bachelorproef}{Structure of this bachelor thesis}}%
\label{sec:opzet-bachelorproef}

% Het is gebruikelijk aan het einde van de inleiding een overzicht te
% geven van de opbouw van de rest van de tekst. Deze sectie bevat al een aanzet
% die je kan aanvullen/aanpassen in functie van je eigen tekst.

De rest van deze bachelorproef is als volgt opgebouwd:

In Hoofdstuk~\ref{ch:stand-van-zaken} wordt een overzicht gegeven van de stand van zaken binnen het onderzoeksdomein, op basis van een literatuurstudie.

In Hoofdstuk~\ref{ch:methodologie} wordt de methodologie toegelicht en worden de gebruikte onderzoekstechnieken besproken om een antwoord te kunnen formuleren op de onderzoeksvragen.

% TODO: Vul hier aan voor je eigen hoofstukken, één of twee zinnen per hoofdstuk
In Hoofdstuk~\ref{ch:proof-of-concept} wordt de toepassing toegelicht die een oplossing op de probleemstelling biedt.

In Hoofdstuk~\ref{ch:conclusie}, tenslotte, wordt de conclusie gegeven en een antwoord geformuleerd op de onderzoeksvragen. Daarbij wordt ook een aanzet gegeven voor toekomstig onderzoek binnen dit domein.