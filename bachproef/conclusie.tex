%%=============================================================================
%% Conclusie
%%=============================================================================

\chapter{Conclusie}%
\label{ch:conclusie}

% TODO: Trek een duidelijke conclusie, in de vorm van een antwoord op de
% onderzoeksvra(a)g(en). Wat was jouw bijdrage aan het onderzoeksdomein en
% hoe biedt dit meerwaarde aan het vakgebied/doelgroep? 
% Reflecteer kritisch over het resultaat. In Engelse teksten wordt deze sectie
% ``Discussion'' genoemd. Had je deze uitkomst verwacht? Zijn er zaken die nog
% niet duidelijk zijn?
% Heeft het onderzoek geleid tot nieuwe vragen die uitnodigen tot verder 
%onderzoek?

Na het uitvoeren van dit onderzoek kan er geconcludeerd worden dat het zeker mogelijk is om met machine learning technologie bedrijfsdocument te analyseren en hieruit relevante masterdata te halen. Om een antwoord te geven op de onderzoeksvraag "Hoe kan machine learning-technologie worden toegepast om relevante masterdata uit bedrijfsdocumenten te halen?" is de proof of concept een voorbeeld van hoe dit gebruikt kan worden, doormiddel van de SAP Business Entity Recognition service.

Het antwoord op de deelvraag 'welke tools bestaan er om dergelijke oplossingen te maken' is terug te vinden in de literatuurstudie in sectie \ref{sec:libraries}. Hieruit is gebleken dat er verschillende tools worden aangeboden, zowel open-source als betalende tools, die als oplossing kunnen dienen om dergelijke applicaties te maken. 

Tijdens het maken van de proof of concept is gebleken dat de BER service van SAP een tool is die verassend gemakkelijk te gebruiken is en die erg goede resultaten geeft. Doordat de BER service weinig configuratie nodig heeft, maakt het een ideale tool om snel een soortgelijke applicatie te maken. Het grootste nadeel van deze tool is dat het niet gratis is wat het minder aantrekkelijk maakt voor kleinere bedrijven.

De proof of concept maken met SAP Fiori was ook een aangename manier om een applicatie te maken. De tools en documentatie die SAP aanbied zijn erg uitgebreid en maken het op die manier erg gemakkelijk om een applicatie te maken. Dit maakt het dan ook in de toekomst mogelijk om de proof of concept verder uit te breiden en te integreren in een bestaande SAP omgeving.